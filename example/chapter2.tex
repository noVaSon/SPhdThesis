\chapter{Chapter 2}
\label{ch:chapter2}

Here ist the link back to chapter \ref{subsec:ch01_structure}.

\section{Section 1}
\label{sec:ch02_section01}

%\parbox[l]{width=0.5\textwidth}{
\begin{wrapfigure}{R}{0.5\textwidth}
\begin{minipage}[l][][l]{0.45\textwidth}
\vspace{-10pt}
\begin{itemize}
  \item Dialog 
  \begin{itemize}
	  \item Production-Sound 
	  \item Add. Dialog Recordings
	  \item Production Effects 
  \end{itemize}
  \item Foley
  \begin{itemize}
	  \item Movement
	  \item Steps
	  \item Props
	  \item Foley Effects
  \end{itemize}
  \item Hard Effects 
  \item Special Effects 
  \item Ambience 
  \begin{itemize}
  	\item Background
  	\item Foreground
  	\item Ambience Effects
  \end{itemize}
  \item Music
  \begin{itemize}
    \item Source Music
    \item Score Music
    \item Music Effects
  \end{itemize}
\end{itemize}%
\vspace{5pt}
\end{minipage}
\captionsetup{width=0.85\linewidth}
\caption[Layers of Film Sound]{Complex film-soundtracks usually apply variations of this layer structure
}
\vspace{-30pt}
\end{wrapfigure}

\paragraph{Special Effects} as a layer contains again multiple layers of
audio clips for each sound effect. A short car crash will be a composition of
screeching wheels, a sharp metal hit sound, a hit on a resonant large
wood box or steel barrel for increasing the physicality and impact and
some shattered glass across the street. Each of these can again be a
mixture of several sounds to create a feeling of the intended sonic
expression to the accompanying picture. This is the most demanding
layer in terms of using pre-recorded sound material. \textit{Sound Effects} are a vertisile category ranging from purely atmospheric and subconsciously effective undefined sound objects like drones to very present and narrative key sounds.
%\pagebreak
\paragraph{Ambience} or \emph{Atmo} is usually a mixture of longer noisy
field-recordings for building an appropriate background for the action
on screen. Also many single cues ("sweeteners") with more
distinguishable sounds are added to enhance the immersion of the film
for example bird calls, wood creaking, distant dog barking, traffic and
so on. This is the second layer where a large collection of sound effect
recordings are necessary to build a good soundtrack. 
\vspace{-11pt}

\paragraph{Foley} can be enhanced or complemented as well by pre-recorded material, perhaps of elements that were difficult to produce in a studio (for example \textit{Hard Effects} like door sounds). Or maybe the low budget does not allow a human foley artist to be payed for his or
her performance, so the editor is in charge of delivering movement and prop
sounds for the scene.

\paragraph{Dialog} needs from time to time a very specific background noise
to stitch gaps between the edges of the edited audio regions that could
be found in a library.

\paragraph{Music} composers frequently use sound effects as well, since the
boundary between sound design and music is fluid, but we will not
concentrate on this exiting topic here. We assume here that composers
and sound designers have the same way of accessing sound effects
collections.

\paragraph{Games and Animations} also have a high demand for sound effects, maybe even
higher, as there is no production sound at all - everything is silent at the
beginning and has to be generated and assigned. In game audio the
mentioned layers are also used, but as the mix is dynamically generated
depending on in-game parameters, this separation often becomes less
obvious in editing projects.

Barbara could certainly write a book of her own. So we will concentrate on the relevant aspects of searching for sounds in her
storage devices.

\pagebreak

\subsection{Subsection 1.1}
\label{subsec:subsection}


\begin{table}[h]
%\caption{Feature frequency among various search related software solutions}
\label{tab:ch04_app_freq}
	\centering
	\begin{tabularx}{\linewidth}{p{0.22\linewidth}||p{0.13\linewidth}|p{0.08\linewidth}|p{0.11\linewidth}|p{0.08\linewidth}|p{0.08\linewidth}|p{0.11\linewidth}}
%		\toprule
		& Spreadsheet view & Grid view & Color mapping & Scatter-plot & Glyphs & Inline Preview\tabularnewline
		\toprule
%		\endhead
		Metadata Management Software & 8 & 1 & 1 & 0 & 0 & 2\tabularnewline
		Interactive Visualization Software & 3 & 5 & 6 & 6 & 3 & 5\tabularnewline
	\bottomrule
	\end{tabularx}
	\caption[Metadata application feature frequency]{Metadata application feature frequency\footnotemark}
	\label{tab:appfeatfreq}
\end{table}
\footnotetext{ you can even have footnotes in captions! See: \ref{fig:ch02_barbara} and \ref{fig:tristimequations} on page \pageref{fig:tristimequations}}


%\pagebreak


