\chapter{chapter 1}
\label{ch:chapter1}

I used the commercial Latex editor Texpad (1.8.9), because I needed to speed up the initial setup time.  To focus on the content and structure I first wrote a short draft with Markup. I was quite in a hurry when I found out, that Markup to Latex conversion is not that gapless as I hoped.

When everything went well you should see the following content:

\begin{itemize}
	\item Title page
	\item \textit{Blank Page}
	\item Acknowledgements
	\item \textit{Blank Page}
	\item Abstract
	\item \textit{Blank Page}
	\item Table of contents
	\item \textit{Blank Page}
	\item List of Abbreviations (nomenclature)
	\item \textit{Blank Page}
	\item List of Figures
	\item \textit{Blank Page}
	\item List of Tables
	\item \textit{Blank Page}
	\item List of source codes (listings)
	\item \textit{Blank Page}
	\item Chapter1
	\item Chapter2
	\item \textit{Blank Page}
	\item Chapter3
	\item References (Bibliography)
	\item Declaration
\end{itemize}

Internal links to other pages of the document should be green, external links (URLs) should appear blue. If you set the option media=print, these colors will be changed to black. You find this line in \textit{thesis.tex}.

To have source code linted (colored depending on input e.g. variables other than functions), have a look at the "minted" package on CTAN \href{http://tug.ctan.org/tex-archive/macros/latex/contrib/minted/minted.pdf}{here}.

\begin{verbatim}
\documentclass[media=screen]{SPhdThesis}
\end{verbatim}

You can speed up rendering by using only placeholders instead of images by adding the "draft" option:

\begin{verbatim}
\documentclass[media=screen, draft]{SPhdThesis}
\end{verbatim}



If you decide to get rid of an unnecessary part of the frontmatter, just comment it out in the main \.tex file: \(thesis.tex\). 

\begin{listing}
\begin{verbatim}
	\begin{frontmatter}
		\begin{acknowledgments}

Thanks everyone

\vspace{1.5cm}
Thank you.
\begin{flushright}

Sincerely yours,\\
\vspace{2cm}
	\SgIntAuthor
\end{flushright} 
\end{acknowledgments} %
		\SgIntClearDoublePage
		\center
\emph{What you can not find in the world, shall be your contribution.}
 
- Inspired by Herrmann Hesse %
		\begin{abstract}

This example thesis is meant to introduce you to a few basic commands that you will need to work with latex. Always look at the source code whenever there is some example, because the interesting stuff happens underneath the surface :)

I want to give you a head-start into writing your thesis with latex, because after some initial hurdles you need to take, it can make the formal stuff so much easier, scientifically more sophisticated an your document will be better readable and even smaller in size.

You find examples of pretty much all functionality that I needed when writing my masters thesis. This template is not written by me, but I added some modifications. Make sure to have a look at the .cls file, because this one is well commented and shows you how to quickly manipulate the look of your whole document with just a few arguments and commands.

Enjoy writing!

\end{abstract}
 %
		\SgAddToc %  Table of contents.
		\SgAddAbbrv % List of abbreviations
		\SgAddLof %  List of figures.
		\SgAddLot %  List of tables.
		\SgAddLos %  List of source codes
		\SgAddLoa %  List of algorithms.
	\end{frontmatter}
\end{verbatim}
\caption{Latex code in the main document \textit{thesis.tex}}
\label{lst:latexexample}
\end{listing}

\section{citation examples}
\label{sec:citeationexamples}

The exact styling used by the citation commands derives from the chosen .bst (bibtex style document).

Always think of proper citation - this is Author year style as defined in the .bst

\begin{verbatim}
	\cite{cooper_about_2014}
\end{verbatim}
Example:	
	\cite{cooper_about_2014}

But you can also only cite the year of a publication 

\begin{verbatim}
	\citeyear{cooper_about_2014}
\end{verbatim} 

Example:
	\citeyear{cooper_about_2014}

There is even a command to put the year in parenthesis 


\begin{verbatim}
	\citeyearpar{cooper_about_2014}.
\end{verbatim} 

Example:
\citeyearpar{cooper_about_2014}

To include a reference in the bibliography without having to cite it in the text, use:

\begin{verbatim}
	\nocite{cooper_about_2014}
\end{verbatim}

	or to include all entries inside the .bib file, even the not cited ones:

\begin{verbatim}
	\nocite*	
\end{verbatim}
\nocite*

% nomenclature example, for List of Abbreviations

Don't forget to feed your nomenclature:
Communication is the key to superior results, think of
computer interface (HCI\nomenclature{HCI}{Human Computer Interface}) aspects.


\section{Motivation}
\label{sec:ch01_motivation}

Always think of people you enjoy.
% 

\begin{wrapfigure}{r}{0.5\textwidth}
\begin{flushright}
\begin{minipage}[t]{0.47\textwidth}
\begin{flushright}
\includegraphics[width=\linewidth]{./pictures/Barb01.png}
\caption[Persona: Barbara]{Barbara:\newline Sound Desiger, Composer, Musician}
\label{fig:ch02_barbara}
\end{flushright}
\end{minipage}
\end{flushright}
\vspace{-12pt}
\end{wrapfigure}


\pagebreak

\section{Structure}
\label{subsec:ch01_structure}

In chapter \ref{ch:chapter2} you find a link back to this chapter.

